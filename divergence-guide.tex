\documentclass{report}
\usepackage{amsmath,amssymb,pxfonts}
\usepackage[utf8]{inputenc}
\usepackage[colorlinks]{hyperref}
\title{Using Divergence and Rebase}
\author{Spencer Tipping}
\begin{document}\maketitle\tableofcontents

\chapter* {Introduction}
   Divergence is a JavaScript library centered around functions and function manipulation. It extends the prototypes of the core JavaScript data types to provide methods that make it easier to
   program in a functional style, and provides a coherent paradigm for generating functions from different types of values.

   Rebase is a Divergence module that decompiles, transforms, and recompiles functions to extend the capabilities of JavaScript. This includes adding syntactic macros, operator overloading, and
   string interpolation to the language by default, and it can also be used as a basis for creating other extensions.

   This guide is intended for anyone who read {\em JavaScript in Ten Minutes}\footnote{Freely available at \url{http://github.com/spencertipping/js-in-ten-minutes}} and would like to see some
   of the ideas put into practice. I think it covers everything, including syntactic macros, continuations, etc. If you haven't yet read {\em JavaScript in Ten Minutes}, I recommend it as
   pre-reading; this guide picks up about where it left off.

   This guide is separated into two parts. The first part goes over the Divergence core library, and the second part introduces Rebase and the ways that it interacts with Divergence.

\section* {Getting Started}
     Divergence and Rebase are both hosted on Github. They can be retrieved from \url{http://github.com/spencertipping/divergence} and \url{http://github.com/spencertipping/divergence-rebase},
     respectively. If you've checked out the user guide repository, then you're all set -- you can open \verb|shell.html| in a non-IE browser and run the examples from there.
    
     If you don't have a local checkout or prefer a more hands-on approach, then you can download \verb|divergence.js| and \verb|divergence.rebase.js| from the Rebase repository, save them
     somewhere, and create an HTML file with the following contents:

\begin{verbatim}
<html>
  <head>
    <script src='divergence.js'></script>
    <script src='divergence.rebase.js'></script>
    <script>
      // Examples can go here
    </script>
  </head>
  <body>
  </body>
</html>
\end{verbatim}

\part {Divergence Core}
\chapter {Everything is a Function}
    The main goal of Divergence is to create a way to convert objects of any basic type into functions. For example, the string \verb|'$0 + $1'| can be promoted into a function that adds its
    first two parameters using the \verb|fn()| method. Similarly, arrays, numbers, booleans, and regular expressions all have defined promotions into functions. I refer to objects that have a
    \verb|fn()| method as {\em functionals}.\footnote{You can write your own functionals, too (see chapter \ref{sec:writing-and-extending-functionals}); Divergence doesn't make any assumptions
    about them other than that they provide {\tt fn()}.}

\section {Booleans and Numbers}
      The promotion patterns for these are simple. If \verb|n| is a number, then \verb|n.fn()| returns a function which returns its \verb|n|$^\textrm{th}$ argument (where arguments' indices are
      zero-based).\footnote{I haven't tried it yet, but I suspect bad things happen if you use floating-point or negative numbers.} So, for example:

\begin{verbatim}
var n = 1;
n.fn() (5, 6)       // => 6
\end{verbatim}

      Because of JavaScript's eager lexing, you can't say things like \verb|5.fn()|; in this case, the dot is considered to be a part of the floating-point literal \verb|5.|. The way to fix this
      is to put a space after \verb|5|, so you have \verb|5 .fn()|. Not quite as nice, but fortunately most function promotion is done implicitly in practice.

      Booleans are promoted as uncurried Church booleans\footnote{Don't worry if this sounds unfamiliar; it's just the theoretical background.} -- that is, \verb|true| returns its first
      parameter and \verb|false| returns its second.

\section {Regular expressions}
      These get promoted into functions that attempt to match the regular expression against a string. For example:

\begin{verbatim}
/foo (bar)/.fn() ('foo bar bif')          // => ['foo bar', 'bar']
/foo (bar)/.fn() ('bar bif baz')          // => null
\end{verbatim}

      Because failed matches return \verb|null|, we have the very nice idiom that \verb|['string1', 'string2', ...].grep(/pattern/)| does exactly what you'd expect.\footnote{{\tt grep} is
      defined for arrays; see chapter \ref{sec:array-functions}}

      Internally the \verb|RegExp.exec()| method is used to achieve this behavior.

\section {Arrays}
      \verb|fn()| distributes across arrays. That is, \verb|[f, g, h].fn()(x)| is equivalent to \verb|[f.fn()(x), g.fn()(x), h.fn()(x)]|. One use of this is to reorder an argument list:

\begin{verbatim}
var g = f.flat_compose ([1, 0]);    // flat_compose implicitly calls fn()
g (x, y)                            // the same as f (y, x)
\end{verbatim}

\section {Functions}
      The \verb|fn()| method exists for functions for the sake of uniformity. It just returns the function.

\section {Strings}
      The \verb|fn()| method for strings does a lot. In the simplest case, it just wraps the string inside an empty function (e.g. \verb|'foo'.fn()| becomes \verb|(function () {return foo})|)
      and runs the result through \verb|eval()|. For example, \verb|'arguments[0]'.fn()| is the identity function.

      Divergence provides a simple regular-expression-based macro processor that does some expansions for you.\footnote{Note that because it just uses regular expressions on the input, there's
      no pretense of protecting things inside strings, etc. Rebase implements a much more sophisticated macro processor that avoids these problems.} The ones that are enabled by default are:

\begin{enumerate}
\item Expressions of the form \verb|$n|, where \verb|n| is an integer, are expanded to \verb|arguments[n]|. For example, \verb|'$0 + $1'.fn()| adds its first two arguments.
\item The expression \verb|$_| is expanded to \verb|this|.
\item The expression \verb|@_| is expanded to \verb|Array.prototype.slice.call(arguments)|.\footnote{This promotes {\tt arguments} into a proper array.}
\item Expressions of the form \verb|@foo|, where \verb|foo| is an identifier, are expanded to \verb|this.foo|.
\item Expressions of the form \verb/{|x, y, z| x + y |}/ are expanded to expressions of the form \verb|(function (x, y, z) {return x + y})|.
\item Expressions of the form \verb|{< expression >}| are expanded to expressions of the form \verb|(function () {return expression})|.
\end{enumerate}

      After performing those substitutions, the result is processed as mentioned before; that is, it's put inside an empty function and \verb|eval|ed.

      Here are some examples:

\begin{verbatim}
'$0[$1]'.fn() ({foo: 'bar'}, 'foo')                       // => 'bar'
'$_ + $0'.fn().call (5, 6)                                // => 11
'$0.map({< $0 + 1 >})'.fn() ([1, 2])                      // => [2, 3]
'$0.fold({|x, y| x * y |})'.fn() ([1, 2, 3, 4, 5])        // => 120
'@foo + @bar'.fn().call ({foo: 10, bar: 5})               // => 15
'@_.length'.fn() (10, 15, 20)                             // => 3
\end{verbatim}

      {\bf Note:} These functions are not closures! For example, if you say something like this:

\begin{verbatim}
var f = function (x) {
  return 'x + $0'.fn();
}
\end{verbatim}

      \noindent you will get unpredictable results, most likely a \verb|ReferenceError| that \verb|x| is undefined. The reason is that \verb|eval()| still obeys lexical scoping rules, which
      means that all of the functions you \verb|eval| through \verb|String.fn| will share the same scope -- the variables in this scope shouldn't really be used by anyone for that reason.
      Referring to globals is probably safe, but anything defined inside a function will be unavailable.

      The way to pass in variables is to preload the function:\footnote{See section \ref{sec:preloading}}

\begin{verbatim}
var f = function (x) {
  return '$0 + $1'.fn(x);
}
\end{verbatim}

      Another important thing to remember is which expansions are function invocations and which are variables. Consider:

\begin{verbatim}
// Important:
'@_.sort(), @_'.fn() (50, 40, 30)                         // => [50, 40, 30]
\end{verbatim}

      The substitutions are purely textual, so any expressions that get generated will be run multiple times and will probably generate unaliased values. \verb|@_| is the only thing that looks
      like a variable but is actually a function call, so I doubt this will be a common source of errors. But it is worth remembering.

      A workaround for the last case, incidentally, in case you need to bind a variable:\footnote{And this isn't one of those cases, as it happens. {\tt sort()} does modify the array in-place,
      but it also returns a reference to the original (which after it returns will be sorted).}

\begin{verbatim}
'{|xs| xs.sort(), xs |} (@_)'.fn() (50, 40, 30)           // => [30, 40, 50]
\end{verbatim}

\chapter {Operations on Functionals}
    \label{sec:operations-on-functionals}

    Functionals, as defined in the previous section, are objects with \verb|fn()| methods. Because they present this interface, we can define operations that work on any functional and add
    them to all of the basic prototypes. All of the following functions are present on every functional.

\section {{\tt compose}}
      \label{sec:compose}
      Composes two functions. Specifically, \verb|f.compose(g)| is equivalent to the function:
      
\begin{verbatim}
function () {
  return f.fn() (g.fn().apply (this, arguments));
}
\end{verbatim}
      
      Notice that both \verb|f| and \verb|g| are automatically promoted into functions. This is true of almost all of the higher-order functions provided by Divergence.

\section {{\tt flat\_compose}}
      \label{sec:flat-compose}
      Composes two functions, but expands the array returned by \verb|g| and supplies the values as arguments to \verb|f|. For example:

\begin{verbatim}
var f = '$0 + $1 * $2';
var g = '[$0, $1, $1]';
f.flat_compose(g) (3, 4)          // => 19
\end{verbatim}

      Specifically, \verb|f.flat_compose(g)| is equivalent to:

\begin{verbatim}
function () {
  return f.fn().apply (this, g.fn().apply (this, arguments));
}
\end{verbatim}

\section {{\tt curry}}
      \label{sec:curry}
      Takes an integer \verb|n| and returns a function that will evaluate the original when called \verb|n| times. For example:

\begin{verbatim}
'$0 + $1 + $2'.curry(1) (1, 2, 3)         // => 6
'$0 + $1 + $2'.curry(2) (1, 2) (3)        // => 6
'$0 + $1 + $2'.curry(3) (1) (2) (3)       // => 6
'$0 + $1 + $2'.curry(4) (1) (2) (3) ()    // => 6
\end{verbatim}

      Arguments don't have to occur in any particular pattern; they're just stuck onto a queue as they're collected. If you want a more traditional implementation of \verb|curry| that chops
      off extras, for example, then \verb|flat_compose| and arrays will probably work:

\begin{verbatim}
var f = '$0 + $1 + $2'.curry(2).flat_compose([0, 1]);
f (1, 2, 3) (4, 5)                        // => 7
\end{verbatim}

      \noindent In this example, the first invocation of \verb|f| took only \verb|1| and \verb|2|. \verb|3| was lost because the array \verb|[0, 1]| doesn't return it anywhere. So at this
      point the queue contains \verb|1| and \verb|2|. On the next invocation, \verb|4| and \verb|5| are passed in, so the queue is now \verb|1, 2, 4, 5|. The function adds the first three
      arguments, totaling 7.

\section {{\tt proxy}}
      This function serves two purposes. One is to get a new function that is extensionally equivalent to, but referentially distinct from, the original function,\footnote{Things that are
      {\em extensionally equivalent} have the same observable behavior, and things that are {\em referentially distinct} refer to different objects. This condition is useful when you want to
      change the state of one object without affecting the other. One particular use case might be assigning a function as a method for multiple classes. Perhaps classes tag the functions,
      e.g.~{\tt method.belongsTo = theClass}. In this case you want the methods to behave the same way but have different attributes.} and the other is to completely intercept the invocation
      of the function.

      Specifically, \verb|f.proxy()| is equivalent to

\begin{verbatim}
function () {return f.fn().apply (this, arguments)}
\end{verbatim}

      \noindent and \verb|f.proxy(g)| is equivalent to (get ready, this is confusing):

\begin{verbatim}
function () {
  var fPrime = f.fn();
  return fPrime.apply.apply (fPrime, g.fn().apply (this, arguments));
}
\end{verbatim}

      Did you catch that? \verb|apply| is itself a function, so we can \verb|apply| it to things. In this case, we apply it to the result of \verb|g|, which is expected to return an array of
      the form \verb|[t, [x1, x2, ...]]| -- \verb|t| refers to the \verb|this| value that \verb|f| should receive, and \verb|x1, x2, ...| are the arguments passed to \verb|f|. This is the
      elephant-gun of composition; most of the time using \verb|compose| or \verb|flat_compose| will do the job.

\section {{\tt bind}}
      The canonical implementation of \verb|bind|, though this one doesn't preload arguments. It also marks the output function with a reference to the original and to the binding, so:

\begin{verbatim}
var o = {foo: 0, bar: 1};
var f = '@foo += @bar + $0';
var g = f.bind(o);
g(5)                      // => 6
g(3)                      // => 10
g.original                // => f
g.binding                 // => o
\end{verbatim}

      The purpose of \verb|bind| is covered in {\em JavaScript in Ten Minutes} and numerous other sources, but the idea is to fix \verb|this| inside a function so that:

\begin{verbatim}
'$_'.fn() ()              // => [object global]
'$_'.bind(5)()            // => 5
\end{verbatim}

      This is not much of an introduction to the concept of function binding; if the purpose or practical use of \verb|bind| is at all unclear, you should definitely read something that goes
      over what it does.

\section {{\tt ctor}}
      The \verb|ctor| function provides a one-step way to initialize the prototype of a function. For example, a quick definition of a 2D vector:

\begin{verbatim}
var vector2 = '@x = $0, @y = $1'.ctor (
              {plus: 'new @constructor(@x + $0.x, @y + $0.y)'.fn(),
                dot: '@x * $0.x + @y * $0.y'.fn()});
var v1 = new vector2 (3, 4);
v1.dot (v1)                       // => 25
\end{verbatim}

      \verb|ctor| takes any number of hashes; they will all be merged together into the prototype of the function.

\section {{\tt type}}
      Prototypes have their advantages and disadvantages. The advantage of performance is paired with the disadvantage of irregular and non-first-class syntax. To illustrate this, consider a
      proxy function for a class constructor; its job is to pass whatever arguments you give it into the constructor for a class:

\begin{verbatim}
var my_class = function () {...};
my_class.prototype.foo = function () {...};
var my_proxy = function () {
  return new my_class ();
};
\end{verbatim}

      This doesn't quite work, since any arguments passed to \verb|my_proxy| are lost. It would be nice to write it this way:

\begin{verbatim}
return new my_class.apply (this, arguments);
\end{verbatim}

      \noindent but that isn't what JavaScript's authors had in mind. As far as I know there is no way around this problem other than using a blank constructor and a first-class initializer:

\begin{verbatim}
var my_class = function () {};                    // This function is empty
my_class.prototype.foo = function () {...};
my_class.initialize = function () {...};          // The real constructor
var my_proxy = function () {
  return my_class.initialize.apply (new my_class (), arguments);
};
\end{verbatim}

      This pattern is abstracted by \verb|type|, which behaves exactly like \verb|ctor| does, except that it uses an empty constructor, uses the given function as the initializer, and returns
      that initializer. An important consequence of this is that the prototype becomes inaccessible by the usual route; instead, you have to create an instance and run
      \verb|x.constructor.prototype| to alter it.\footnote{In the future I may introduce an option to make the prototype visible.}

\section {Preloading}
      \label{sec:preloading}
      A feature of the \verb|fn()| method I haven't mentioned is that you can use it to preload arguments to a function. For example:

\begin{verbatim}
var f = '$0 + $1'.fn (1);
f (2)                     // => 3
f (3, 4, 5)               // => 4

var g = '@_.length'.fn (3, 4, 5);
g (1, 2)                  // => 5
g (1)                     // => 4
g ()                      // => 3
\end{verbatim}

      This feature is present for all of the \verb|fn()| methods, not just on the one for strings:

\begin{verbatim}
var f = [0, 1, 2].fn ('foo', 'bar');
f ('bif')                 // => ['foo', 'bar', 'bif']
\end{verbatim}

\chapter {Delimited Continuations}
    \label{sec:delimited-continuations}

    In addition to the operators in chapter \ref{sec:operations-on-functionals}, Divergence includes two more to make it easy to encode delimited continuations in CPS code.\footnote{If the
    terms ``delimited continuations,'' ``continuation-passing style,'' and ``{\tt call-with-current-continuation}'' are unfamiliar, then you should probably skip this chapter.} The idea is
    that you can have a localized region of CPS code which then calls an escaping continuation that returns normally. It must escape eventually, but Divergence provides a continuation encoding
    that lets you run CPS-converted code with tail call optimization.

    For example, consider a regular recursive factorial function:

\begin{verbatim}
var fact = function (n) {
  return n > 1 ? n * fact (n - 1) : n;
};
fact (5)                    // => 120
\end{verbatim}

    This will chew through $O(n)$ stack frames. A tail-recursive encoding would be better:

\begin{verbatim}
var fact = function (n, acc) {
  return n > 1 ? fact (n - 1, acc * n) : acc;
};
fact (5, 1)                 // => 120
\end{verbatim}

    \noindent but because most JavaScript engines don't perform tail-call optimization it has the same problem. This is where you can use the \verb|cps()| and \verb|tail()| methods on
    functionals to write a constant-stack-space solution:

\begin{verbatim}
var fact = function (n, acc, k) {
  return n > 1 ? fact.tail (n - 1, acc * n, k) : k.tail (acc);
};
fact.fn(5, 1).cps('$0')     // => 120
\end{verbatim}

    The arguments get preloaded onto \verb|fact()|, and \verb|cps| takes the final continuation.\footnote{You can also omit the final continuation, in which case it just returns its first
    argument.} When that is called, \verb|cps| returns its value. When you call a function in CPS mode, it should return \verb|tail| invocations on other functionals. Two things are important
    here:

\begin{enumerate}
\item CPS-mode functions must invoke a continuation using a \verb|tail| call.
\item The invocation of \verb|tail| must be returned. (Thus ensuring that it is in fact a tail call.)
\end{enumerate}

\section {Implementation}
      Encoding delimited continuations is actually not a difficult matter assuming that the language supports tail-call optimization. The real problems are (1) coming up with a notation that
      looks vaguely like the formal \verb|shift| and \verb|reset|,\footnote{You'll notice that I've completely chickened out here. You have to write CPS-converted code up front, which is quite
      lame.} and (2) finding some way of delegating to the next continuation that doesn't eat stack frames. Divergence provides this second point with the \verb|cps| and \verb|tail| functions.

      When you perform a tail call, you could think of it as merging another stack frame into the current one; that is, the current stack state gets overwritten and the new variables replace
      it. JavaScript doesn't provide a way to do this, but it does provide a way to get rid of a stack frame altogether. The \verb|return| statement that's mandatory for \verb|tail| calls does
      exactly what it looks like, just in a different order:

\begin{verbatim}
return fact.tail (n - 1, acc * n);        // returns [fact, [n - 1, acc * n]]
\end{verbatim}

      The \verb|cps| function runs a \verb|while| loop that waits for return values of this form. When it finds one, it simply calls \verb|apply|:

\begin{verbatim}
while (result[0] !== c)
  // tail call really happens here:
  result = result[0].apply (this, result[1]);
\end{verbatim}

      Finally, when it sees a tail call to the final continuation, it breaks out of the \verb|while| loop and returns to regular (non-CPS) code:

\begin{verbatim}
return c.apply (this, result[1]);
\end{verbatim}

\chapter {Array Functions}
    \label{sec:array-functions}
    Divergence adds some useful methods to arrays. It's mostly the usual functional stuff:

\begin{enumerate}
\item[{\tt map}]      Maps a function across the elements of an array and returns a new array of the results. For example:

\begin{verbatim}
[1, 2, 3].map ('$0 + 1')                        // => [2, 3, 4]
\end{verbatim}

\item[{\tt grep}]     Returns an array of the elements for which a function returns a true-ish value:

\begin{verbatim}
[1, 2, 3, 4, 5].grep ('$0 % 2')                 // => [1, 3, 5]
\end{verbatim}

\item[{\tt fold}]     Left-reduces an array under a binary operation. Can take additional arguments for preloading:

\begin{verbatim}
[1, 2, 3].fold ('$0 + $1')                      // => 6
['b', 'c', 'd'].fold ('$0 + $1', 'a')           // => 'abcd'
\end{verbatim}

\item[{\tt sort\_by}] Sorts an array through a projection; that is, returns the original values, but sorted depending on the output of some other function. The function you provide will
                        probably be called $O(n \log n)$ times.

\begin{verbatim}
['a', 'bc', 'def'].sort_by ('- $0.length')      // => ['def', 'bc', 'a']
\end{verbatim}

\item[{\tt flat\_map}] Just like \verb|map|, except the results are assumed to be arrays and are concatenated together:

\begin{verbatim}
[1, 2, 3].flat_map ('[$0, $0 + 1]')             // => [1, 2, 2, 3, 3, 4]
\end{verbatim}

\item[{\tt each}]     Just like \verb|map|, but doesn't store the output of the function (the original array is returned instead). This is slightly more efficient because a second array
                        isn't allocated.
\end{enumerate}

\chapter {Writing and Extending Functionals}
    \label{sec:writing-and-extending-functionals}

    The definitions of functional operations such as \verb|compose|, \verb|ctor|, etc.~are the same across prototypes. That is, \verb|Number.prototype.ctor| is the same function as
    \verb|Function.prototype.ctor|, which is the same as \verb|String.prototype.ctor|. This makes it very easy to add your own functional operations.

\section {Extending functionals}
      Divergence provides the \verb|d.functions()| method for extending the set of operations on functionals. For example, here is how you might go about defining a method to pluralize a
      functional:

\begin{verbatim}
d.functions ({pluralize: '{|f, xs| xs.map(f) |}.fn (@fn())'.fn()});
'$0 + 1'.pluralize() ([1, 2, 3])          // => [2, 3, 4]
\end{verbatim}

      \verb|d.functions| uses \verb|d.init| internally,\footnote{See chapter \ref{sec:miscellaneous-other-stuff}} so you can pass in multiple arguments, and each argument can either be a hash
      or a function.

      To see what extensions have already been made for functionals, you can reference \verb|d.functional_extensions|:

\begin{verbatim}
d.functional_extensions.compose           // => [the compose function]
\end{verbatim}

      Note, however, that modifications to \verb|functional_extensions| will have no effect on operations that are made available through the basic prototypes. You should always use
      \verb|d.functions()| to define new operations.
    
\section {New functionals}
      Let's suppose that we want to write a functional to compose two existing ones. Obviously, in order to be a functional at all it will have to have \verb|fn()|, and in order to be
      constructable it will need to have a type of some sort. Here's a start then:

\begin{verbatim}
var composition = '@lhs = $0, @rhs = $1'.ctor ({
  fn: '{|f, xs| f.fn.apply (f, xs) |} (@lhs.compose(@rhs), @_)'.fn()});
\end{verbatim}

      This gets us \verb|fn()|, but what about all of the default goodies that come with functionals? Divergence provides a method just for this purpose:

\begin{verbatim}
d.functional (composition);
\end{verbatim}

      This has two effects. First, it adds all of the methods defined in \verb|d.functional_extensions| to \verb|composition|, and second, it pushes \verb|composition| onto
      \verb|d.functionals|.\footnote{{\tt d.functionals} is an array of things to extend later on when we call {\tt d.functions}. {\tt d.functional\_extensions} is a member of this array;
      that's why it stays up-to-date.} The net effect of this and \verb|d.functions| is that provided that you use this API, all of your functionals will be kept in the same state.

\chapter {Writing Inline Macros}
    This is actually really easy. You just use the \verb|macro()| method on \verb|RegExp|, like this:

\begin{verbatim}
/\$(\w+)/.macro ('"arguments[0]." + $1'.fn());
'$foo'.fn()                 // => (function () {return arguments[0].foo})
\end{verbatim}

    There are a couple of things to remember here. One is that you must use \verb|fn()| on the expander, since it would also be legitimate to replace a word with a constant string. The other
    is that your function is given as the second argument to the \verb|String.replace()| method, so the first parameter to your function will be the entire match range. (This results in the
    rather Perlish property that \verb|$1| is the first match, \verb|$2| the second, etc.)

    Also, macros that you define will be run {\em after} all of the previously-defined macros. This means two things. First, patterns that you define can't override patterns that get
    transformed by other macros (unless you're prepared to transform their output), and second, your macroexpansions can't rely on macro shortcuts.\footnote{Because of these unfortunate
    limitations, I may change {\tt macro()} to prepend macro definitions in the future.}

    Divergence also includes a mechanism to expand macros without evaluating the result. You can call the \verb|d.macro_expand| on a string to see what will be put inside a function:

\begin{verbatim}
d.macro_expand('$_')                // => 'this'
d.macro_expand('{|x| x + 1 |}')     // => '(function(x){return  x + 1 })'
\end{verbatim}

\chapter {Miscellaneous Other Stuff}
    \label{sec:miscellaneous-other-stuff}
    Aside from the core functions, there are some other useful things that Divergence provides. One of them is the \verb|d.map| function, which iterates over the key-value pairs of a hash.
    It's a monadic bind over hashes (analogous to a flat map), so your function returns some hash and at the end they're all merged together to form the result. For example:

\begin{verbatim}
d.map ({foo: 'bar', bif: 'baz'}, '$0.maps_to($1 + '1')')
    // => {foo: 'bar1', bif: 'baz1'}

d.map ({foo: 'bar'}, 'd.init ($1.maps_to($0), $0.maps_to($1))')
    // => {foo: 'bar',  bar: 'foo'}

d.map ({foo: 'bar', bif: 'baz'}, '/f/.test($0) && $0.maps_to($1)')
    // => {foo: 'bar'}
\end{verbatim}

    Another is the \verb|d.init| function, which takes an object and a series of modifiers, applies the modifiers to the object, and returns the original. The modifiers can be hashes, in which
    case their values are merged onto the object, or they can be functions, in which case they are applied to the original with \verb|this| equal to the object. For example:

\begin{verbatim}
var f = d.init (function () {return 5}, {returns: 5});
f.returns                           // => 5
f ()                                // => 5
var g = d.init ('$0 + 1'.fn(), {throws: null}
  '@returns = $0, @inverse = $1'.fn ('x plus 1', '$0 - 1'.fn()));
g.returns                           // => 'x plus 1'
g.inverse (5)                       // => 4
g.throws                            // => null
\end{verbatim}

    The initializers are applied in the order provided.

    These are probably the two most useful global functions in the Divergence core. There are a few others, though; I recommend looking over the source code (it isn't long, well under 100
    lines total) to get a feel for how it works internally. In particular, the builtin macro definitions can be extended; the source code provides examples of how to do this.

\part {Rebase}
\chapter {Getting Started with Rebase}
    Rebase decompiles, transforms, and recompiles functions to provide low-level rewriting of JavaScript's core constructs. (Hence the name.) This process is complex and not for the faint of
    heart; fortunately, the interface is relatively straightforward. For example:

\begin{verbatim}
var f = function () {...};          // A normal function
var g = d.rebase (f);               // g is a transformed copy of f
\end{verbatim}

    There are, however, a few things to keep in mind:

\begin{enumerate}
\item Some information is lost during decompilation. Particularly, functions that close over variables will lose their closure bindings, since JavaScript provides no way to access them in a
        first-class way. So all calls to \verb|d.rebase| should occur at the top-level -- that is, outside of all other function bodies.\footnote{Alternatively, you can use {\tt
        d.rebase.local} with {\tt eval}, which uses {\tt eval}'s dynamic scoping to keep the scope chain intact.}

\item Transformed or promoted functions can't be rebased directly. Rebasing calls \verb|toString()| on a function to obtain its code, and functions that proxy application don't generally
        also proxy \verb|toString()|. So, whenever you're rebasing a function, I recommend passing in the function directly:

\begin{verbatim}
d.rebase (function () {
  ...
});
\end{verbatim}
\end{enumerate}

    The following chapters discuss things you can do inside a rebased function, and how that code gets recompiled into normal JavaScript.

\chapter {Operator Overloading}
    Rebase lets you overload JavaScript's operators. In practice, this entails translating operator invocations into method calls.\footnote{Some operators are not translated because they cause
    behavior to change. These include {\tt ==}, {\tt ===}, {\tt !=}, {\tt !==}, {\tt =}, {\tt ++}, {\tt --}, {\tt \&\&}, {\tt ||}, {\tt ?:}, function calls, dot-lookups, hash-lookups, commas,
    and {\tt !}.} For example, this function:

\begin{verbatim}
function (x, y) {return x + y}
\end{verbatim}

    \noindent would be rebased into:

\begin{verbatim}
function (x, y) {return x['+'](y)}
\end{verbatim}

\section {Standard operators}
      Because this transformation also affects regular values such as numbers and strings, Rebase installs handler functions for these objects. If you look at \verb|String.prototype|, for
      instance, after Rebase is included, you'll probably see a bunch of functions whose names are operators; these are compatibility functions that just delegate to those operators to provide
      normal operation after the operators have been converted.

      Rebase also provides some default operators in places where JavaScript's defaults aren't very helpful:

\begin{verbatim}
[1, 2, 3] * (function (x) {return x + 1})     // => [2, 3, 4]
[1, 2, 3] % (function (x) {return x % 2})     // => [1, 3]
[1, 2, 3] + [4, 5, 6]                         // => [1, 2, 3, 4, 5, 6]
[1, 2, 3] / (function (x, y) {return x + y})  // => 6
[1, 2, 3] >>$- f                              // => [1, 2, 3].flat_map (f)
\end{verbatim}

      These operators (for arrays, anyway) are aliased to regular methods; \verb|*| is aliased to \verb|map|, \verb|/| to \verb|fold|, \verb|%| to \verb|grep|, \verb|+| to \verb|concat|, and
      \verb|>>$-| to \verb|flat_map|.\footnote{{\tt map}, {\tt grep}, {\tt flat\_map}, and {\tt fold} are provided by the Divergence core library. {\tt concat} is a standard array method.} The
      \verb|>>$-| notation comes from monadic binding and was chosen because it looks vaguely like Haskell's \verb|>>=|.\footnote{Because $\gg=$ is an assignment operator in JavaScript, the
      left-hand side must be a proper lvalue. This is enforced within the JavaScript grammar, so had I used this notation all monadic binding would have do be done against unadorned
      variables.}

      At this point you might reasonably ask about \verb|>>$-|; this is certainly not a normal-looking operator! Quite right -- Rebase allows you to combine binary operators to form new compound
      ones around certain identifiers. \verb|$| is one such identifier (these are called ``sandwiches''), and the rule is that if the compiled expression tree includes two adjacent binary
      operators around a sandwich identifier, then they get merged to form something like \verb|>>$-|. Section \ref{sec:sandwiches} goes over this in more detail.

\section {Defining new operators}
      It's actually very simple to define new operators for your classes, or to redefine the ones that Rebase defines for standard classes. Here is how you might go about defining a 2D vector,
      for instance:

\begin{verbatim}
var vector2 = '@x = $0, @y = $1'.ctor ({
  toString: function () {return '<' + this.x + ', ' + this.y + '>'},
       '+': 'new @constructor(@x + $0.x, @y + $0.y)'.fn(),
       '-': 'new @constructor(@x - $0.x, @y - $0.y)'.fn(),
       '*': 'new @constructor(@x * $0.x, @y * $0.y)'.fn(),
       '/': 'new @constructor(@x / $0.x, @y / $0.y)'.fn()});
\end{verbatim}

      An equivalent and more compact way to do this using Rebase function literals and string interpolation:

\begin{verbatim}
var vector2 = '@x = $0, @y = $1'.ctor (
  ([{toString: _ >$> '<#{this.x}, #{this.y}>'}] +
   '+ - * /'.split(' ') * (op >$> op.maps_to (
     'new @constructor (@x #{op} $0.x, @y #{op} $0.y)'.fn()))) / d.init);
\end{verbatim}

      Now you can use these operators:

\begin{verbatim}
d.rebase (function () {
  var v1 = new vector2 (3, 4);
  var v2 = new vector2 (1, 6);
  alert (v1 + v2);          // Alerts '<4, 10>'
}) ();
\end{verbatim}

      You can define compound operators the same way:\footnote{Some compound operators are macros and will never be run. See section \ref{sec:builtin-macros} for more details.}

\begin{verbatim}
vector2.prototype['-$*'] = '@x * $0.x + @y * $0.y'.fn();
d.rebase (function () {
  var v1 = new vector2 (3, 4);
  alert (v1 -$* v1);        // Alerts '25'
}) ();
\end{verbatim}

\section {Sandwiches}
      \label{sec:sandwiches}
      Earlier I mentioned the \verb|>>$-| operator, which is really a combination of \verb|>>|, \verb|$|, and \verb|-|. The conditions required for rebase to sandwich these tokens into one
      are:

\begin{enumerate}
\item \verb|d.rebase.sandwiches| must map \verb|$| to a true-ish value (which it does by default)
\item \verb|d.rebase.sandwich_ops| must map \verb|>>| and \verb|-| to true-ish values (which it also does)
\item If we were interpreting the script normally, the evaluations of \verb|>>| and \verb|-| would have to be adjacent.
\end{enumerate}

      This third point deserves some explanation. When Rebase is going through your code, it first lexes into tokens, then parses into an expression tree, then transforms that expression tree
      with any macros that are defined (see chapter \ref{sec:macros}), and finally serializes and \verb|eval|s the result.

      The operator sandwiching doesn't happen at the lexing stage, however. It's implemented as a transformation of the expression tree, which means that some information has been lost. In
      particular, it is unclear whether you typed \verb|x + y >>$- z| or \verb|x + (y >> $) - z|.\footnote{Since Rebase keeps track of the original parens, this actually wouldn't normally be a
      problem. The issue arises when serializing a function through SpiderMonkey; this JS engine parses a function into a parse tree and does constant folding before serializing it via {\tt
      toString}. Unfortunately, this means that a considerable amount of information has been lost.} To avoid mangling the second case, Rebase is conservative about which operators it
      replaces; thus the restriction that \verb|>>| and \verb|-| must have a direct parent-child relationship in the parse tree.\footnote{This also makes it less feasible to define sandwiched
      operators whose left and right components have very different precedence. Because of the way SpiderMonkey presents functions, I think this is ultimately a good thing to be aware of.}

\chapter {Macros}
    \label{sec:macros}
    Just like Divergence's inline macro processor, Rebase maintains a list of transforming functions not on strings but on expression trees; these are stored in \verb|d.rebase.macros|. When
    you call \verb|d.rebase(function)|, here is what happens:

\begin{enumerate}
\item \verb|d.rebase.parse(function.toString())| is called, which translates the function into an expression tree.\footnote{Note that because we're just calling {\tt toString()}, but never
        actually applying the argument to {\tt d.rebase.parse}, you could in fact rebase just about anything. This includes expression trees, strings, or anything else that has a
        {\tt toString()} method that produces parseable code.}
\item The expression tree is traversed depth-first, and each node is folded over all of the macros; that is, \verb|node = macros.fold('$0($1)', node)|. This has the effect of composing all
        of the macros together, so that if \verb|[m1, m2, ..., mn]| is the macro list, then \verb|node| will become \verb|mn(...(m2(m1(node)))...)|.
\item \verb|node.toString()| is called on the top tree node (Rebase generates this; it is always \verb|(value)|).
\item The result is \verb|eval|ed and that output is returned.
\end{enumerate}

    The difference between \verb|d.rebase(f)| and \verb|d.rebase.local(f)| is that \verb|d.rebase.local(f)| leaves off the final \verb|eval| step. This lets you call \verb|eval| directly to
    change the surrounding scope. (Note that this has performance implications.)

\section {Built-in macros}
      \label{sec:builtin-macros}
      The comments in the Rebase source code go over the mechanics of implementing the built-in macros, but here is what they do:

\begin{enumerate}
\item The first macro performs operator sandwiching. This has to be run before we expand assignments, since otherwise you might have unintended effects from statements such as
          \verb|x += $ >> 5|.

\item The next macro to be run is the assignment-expander. This takes expressions of the form \verb|x += y|, \verb|x <<= y|, etc.~and translates them into their full forms,
          e.g.~\verb|x = x + y| or \verb|x = x << y|. This is actually necessary to preserve behavior. The reason is that the left-hand side of \verb|+=| or any other assignment operator must
          be an lvalue, and \verb|this| isn't an lvalue.\footnote{Lvalues are things that can be assigned to. I think the terminology comes from the fact that you can put them on the left-hand
          side of an assignment operator. Anyway, JavaScript lets you assign to variables and hash and array entries, and that's about it.} The problem becomes apparent when we want numbers to
          preserve their behavior:

\begin{verbatim}
Number.prototype['+='] = '$_ += $0'.fn();     // Can't do this
\end{verbatim}

          The simplest answer is just to expand all of these expressions and forgo whatever operator overloading we might have been able to achieve with them (not a lot as it happens, since
          the JavaScript grammar is quite restrictive about what you can use as an lvalue).

\item Inline functions. Rebase provides a syntax for defining functions that is a bit more lightweight than JavaScript's \verb|function () {return x}| syntax. Instead, you can use the
          infix operator \verb|>$>|, like this: \verb|x >$> x + 1|. Its precedence is with relational operators, so this code will do what you expect:

\begin{verbatim}
var f = x >$> x + 1;
var g = y >$> y << 5;
\end{verbatim}

          Anything at or below a relational operator, however, must be parenthesized:

\begin{verbatim}
var f = (x, y) >$> (x !== y);
var g = (x, y, z) >$> (x ? y : z);
\end{verbatim}

          Unfortunately, JavaScript won't let you define a nullary function as \verb|() >$> x|. However, you can bind a throwaway variable such as \verb|_| and use that instead:
          \verb|_ >$> x|. Since JavaScript doesn't track formal parameters anyway, there isn't much difference.

          Note that this macro transforms expressions of the form \verb|args >$> expression| into \verb|(function (args) {return expression})|. This has some important consequences, perhaps
          foremost that \verb|this| and \verb|arguments| take on different meanings on the right-hand side of \verb|>$>|. So, for example, this function will not do what seems obvious:

\begin{verbatim}
String.prototype['*'] = f >$> (
  this.split('').map(x >$> f(x, this)));
\end{verbatim}

          The inner \verb|this| that gets passed to \verb|f| will be \verb|[object global]|, not the original string.

\item Function preloading. Because there isn't a good way to bind variables in expression mode, Rebase uses inline functions for this purpose. To localize the variable name with its value,
          you can use the \verb+|$>+ operator:

\begin{verbatim}
5 |$> (x >$> x + 1)   // => 6
\end{verbatim}

          This code gets translated into:

\begin{verbatim}
(x >$> x + 1) (5)
\end{verbatim}

          Note that you don't have to use \verb+|$>+ with inline functions; it can also be used with named functions:

\begin{verbatim}
'foo' |$> document.getElementById     // preserves binding to document
\end{verbatim}

          You can preload multiple arguments at once, too:

\begin{verbatim}
(5, 6) |$> ((x, y) >$> x + y)         // => 11
\end{verbatim}

          Unfortunately, due to JavaScript's syntax, you can't perform nullary invocation:

\begin{verbatim}
() |$> f      // syntax error
\end{verbatim}

          However, none of the use-cases for preloading functions involve nullary application.

\item Comment processing. Rebase supports structural commenting, which lets you remove things on an expression basis and replace them with \verb|undefined|. For example:

\begin{verbatim}
var f = x && comment(y + z);
\end{verbatim}

          If you run this code, \verb|y + z| will never be evaluated (in fact, it won't even appear in the resulting function). The code that will be generated looks like this instead:

\begin{verbatim}
var f = x && undefined;
\end{verbatim}

\item String interpolation. This is one of my favorites because it's so useful. Rebase will go through your program and expand every string with \verb|#{...}| segments into a string
          concatenation. For example:

\begin{verbatim}
var s = 'The number is #{3 + 5}';
\end{verbatim}

          \noindent is expanded into:

\begin{verbatim}
var s = ('The number is ' + (3 + 5) + '');
\end{verbatim}

          Code inside these escapes is rebased, so you can also say things like this:

\begin{verbatim}
var s = 'The number is #{(x >$> x + 1) (5)}';
\end{verbatim}

          Strings without these sequences are left alone. This is necessary to preserve the integrity of hash-keys, which must be unparenthesized literal strings or identifiers.

\item The last thing that happens is operator overloading. Once we've translated everything, we replace all binary operators\footnote{With some exceptions -- see {\tt
          d.rebase.should\_convert}, for instance.} with method calls. Precedence is preserved, so you get nested expressions of the form \verb|x['+'](y['*'](z))|, for instance.
\end{enumerate}

\section {{\tt literal}}
      Sometimes you want to protect a piece of code from any kind of macro transformation. One reason for this is for performance; another reason might be to work around a bug that pops up due
      to some incompatibility in the Rebase standard functions. Fortunately, this is quite easy using the \verb|literal| keyword:

\begin{verbatim}
d.rebase (function (y) {
  var x = literal (y + 5) + 6;
});
\end{verbatim}

      Macro transformation, including string interpolation, operator overloading, and everything else, won't enter a \verb|literal| expression. So the code above translates into:

\begin{verbatim}
function (y) {
  var x = (y + 5)['+'](6);
}
\end{verbatim}

\section {Working with syntax trees}
      Syntax trees have a simple API used for incremental parsing, but most likely you won't want to use it for anything serious. Let's suppose you want to write a macro to expand to a caching
      function, something that transforms this:

\begin{verbatim}
$|| x + y
\end{verbatim}

      \noindent into this:

\begin{verbatim}
(function (value, computed) {
  return function () {
    return computed ? value : (value = (x + y));
  };
}) (null, false)
\end{verbatim}

      There are a bunch of syntax nodes there and it would be a major pain to construct them all. The simplest way to write this macro would be:

\begin{verbatim}
function (e) {
  return    e.op == '||' &&
         e.xs[0] ==  '$' ?
      d.rebase.parse ('__f__ (null, false)').find ('__f__')[0].
        replace (0,
          d.rebase.parse (function (value, computed) {
            return function () {
              return computed ? value : (value = __expression__);
            };
          }).find ('__expression__')[0].replace (0, e.xs[1]).top()).top() : e;
}
\end{verbatim}

      I don't expect this makes sense, so I'll go through it. First we construct the function invocation as a quoted string. It needs to be a string, since normal expressions get evaluated
      eagerly before Rebase can see them. We have Rebase convert it to a syntax tree using \verb|parse()|, then find the \verb|__f__| node and have it replace its first child with a new tree.
      The new tree here is a function whose \verb|__expression__| is replaced by the right-hand side of the original \verb+||+ operator in the \verb+$||+ expression.
\end{document}
